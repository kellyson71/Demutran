\chapter{Metodologia}

Utilizei técnicas de manutenção preventiva e corretiva, envolvendo análise de hardware, substituição de peças e otimização de sistemas. Baseei-me em diagnósticos constantes e na análise de desempenho dos equipamentos para identificar e solucionar problemas.

\section{Tecnologias e Ferramentas Utilizadas}

Durante o estágio, trabalhei com diversas ferramentas essenciais para garantir o bom funcionamento dos sistemas. Para suporte remoto, utilizei principalmente o AnyDesk, que permitiu atender as demandas dos usuários de forma rápida e eficiente. 

No diagnóstico de hardware, o CPU-Z foi fundamental para análise detalhada dos componentes do sistema, enquanto o CrystalDiskInfo auxiliou no monitoramento da saúde dos discos rígidos. Para testes de memória RAM, o Memtest86 se mostrou uma ferramenta confiável na identificação de possíveis falhas.

Para otimização e manutenção dos sistemas, o CCleaner foi utilizado regularmente na limpeza e organização dos computadores. Complementando esse trabalho, o Driver Booster garantiu que todos os drivers permanecessem atualizados, prevenindo problemas de compatibilidade e melhorando o desempenho geral dos equipamentos.

\section{Procedimentos Adotados}

\subsection{Manutenção de Equipamentos}

Durante o período de estágio, desenvolvi uma rotina sistemática de manutenção dos equipamentos. Realizava verificações periódicas nos computadores, identificando e resolvendo problemas antes que afetassem o trabalho dos usuários. Nas impressoras, além da limpeza regular, aprendi a identificar peças desgastadas e realizar pequenos reparos, prolongando sua vida útil e reduzindo custos com manutenção externa.

\subsection{Implementação de Cabeamento de Rede}

Uma das experiências mais enriquecedoras foi trabalhar com a infraestrutura de rede. Aprendi na prática sobre diferentes categorias de cabos e suas aplicações específicas. Durante a organização do cabeamento estruturado, pude aplicar técnicas de crimpagem seguindo o padrão T568B, além de desenvolver habilidades para organizar os cabos de forma eficiente, facilitando futuras manutenções e melhorando significativamente a qualidade da rede local.

\subsection{Suporte Técnico}

O contato direto com os usuários foi um dos aspectos mais gratificantes do estágio. Desenvolvi um sistema próprio para registrar e acompanhar as solicitações, priorizando casos mais urgentes. A experiência me ensinou não apenas a resolver problemas técnicos, mas também a me comunicar de forma clara e paciente, garantindo que os usuários se sentissem seguros e bem atendidos. Esta abordagem resultou em feedbacks positivos e em um ambiente de trabalho mais produtivo.

